\documentclass[a4paper,12pt]{article}

\usepackage[top=3cm, bottom=2cm, left=3cm, right=2cm]{geometry}
\usepackage[utf8]{inputenc}
\usepackage[portuguese]{babel}
\usepackage{booktabs}
\usepackage{multirow}
\usepackage{multicol}
\usepackage{graphicx}
\usepackage{longtable}
\usepackage{verbatim}
\usepackage{minted}

\usepackage{float}
\floatstyle{ruled}
\newfloat{program}{thp}{lop}
\floatname{program}{Script}

\title{Tecnologias de Base de Dados\\
\Large{Trabalho de BD e XML, usando PL/SQL na Web}}

\author{André Fernandes (ei03107) \and Pedro Batista (ext10392)}

\begin{document}

\maketitle

\section{Obtenha uma representação em XML da situação descrita.}
\subsection{Defina a linguagem com um DTD apropriado, baseado na descrição da
situação e no modelo relacional e colocando-se na perspectiva dos médicos. O
resultado deve ter alguma flexibilidade e incluir elementos e atributos.}

O DTD abaixo foi definido de modo a representar a clinica, observa-se que os
dados estão ligados principalmente ao médico, com exceção do cliente, pois se
este fosse associado diretamente a consulta implicaria numa grande redundância
de dados.

Outra especificidade do DTD apresentado é que a \emph{situacao} na
\emph{consulta} foi modelada como atributo pois só assim é que podemos garantir
que seu valor está no intervalo requerido.

\inputminted{xml}{clinica.dtd}

\subsection{Construa um documento instância do DTD acima que contenha pelo menos a
informação das linhas a amarelo. Sugere-se a utilização de queries à BD para
esse efeito.}

A seguinte query foi elaborada de forma a retornar um documento instancia do DTD
com todos os dados contidos na base de dados. É importante destacar, que nessa
query os identificadores do médico e do cliente são modificados de forma introduzir
um sufixo \emph{m} e \emph{c} respectivamente. Isto é necessário pois de acordo
com a definição, um identificador XML deve começar por uma letra.

O resultado da query é muito longo, por isso o XML resultante é apresentado em
anexo na Seção~\ref{anex:xml_clinica}. Este foi validado com o DTD utilizando a
biblioteca \emph{The XML C parser and toolkit of
Gnome}\footnote{http://xmlsoft.org/} (libxml) especificamente utilizando o programa
\emph{xmllint}\footnote{http://xmlsoft.org/xmllint.html} com o comando:
\verb~xmllint --dtdvalid clinica.dtd --noout clinica.xml~

\inputminted{sql}{clinica.sql}

\section{Utilizando XSL sobre XML sobre SQL implemente os seguintes
processamentos:}

\subsection{Mostrar o nome e a data de nascimento dos oftalmologistas.}

O seguinte XSL quando aplicado ao XML apresentado na
Seção~\ref{anex:xml_clinica} transforma-o em uma tabela XHTML apresentando a
data de nascimento e o nome dos oftalmologistas. O programa foi testado no
navegador Firefox 4 e na biblioteca \emph{The XSLT C library for
GNOME}\footnote{http://xmlsoft.org/XSLT/} (libsxlt) com o 
seguinte comando: \verb~xsltproc clinica.xml~ o resultado é apresentado em anexo
na Seção~\ref{anex:dt_oftalmo.xhml}.
\inputminted{xml}{dt_oftalmo.xsl}

\subsection{Mostrar o número e o nome dos médicos numa lista e, dentro de cada
item, apresentar uma tabela com o dia e hora das respectivas consultas, com o
nome do doente e o relatório.}

O seguinte XSL também pode ser aplicado a instancia XML da
Seção~\ref{anex:xml_clinica}, este produz em uma lista XHTML o nome dos médicos
e em uma tabela o dia e a hora de suas consultas, este foi testado no navegador
Firefox 4. O resultado utilizando a libxslt é apresentado em anexo na
Seção~\ref{anex:medico_consulta.xhtml}.
\inputminted{xml}{medico_consulta.xsl}

\newpage 
\section{Anexo}

\subsection{XML representando os dados da base de dados}\label{anex:xml_clinica}
\inputminted{xml}{clinica.xml}

\subsection{Transformação do XML apresentado na Seção~\ref{anex:xml_clinica} de
forma a mostrar a data de nascimento e o nome dos médicos
oftalmologistas}\label{anex:dt_oftalmo.xhml}
\inputminted{xml}{dt_oftalmo.xhtml}

\subsection{Transformação do XML apresentado na Seção~\ref{anex:xml_clinica}
para apresentar as consultas marcadas para cada
médico}\label{anex:medico_consulta.xhtml}
\inputminted{xml}{medico_consulta.xhtml}

\end{document}
